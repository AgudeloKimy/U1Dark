\cleardoublepage

\chapter{Archivos de SARAH para el modelo $U(1)_D$}
\label{AppendixE}
\lhead{Apéndice E. \emph{Archivos de SARAH para el modelo U(1) oscuro}} % This is for the header on each page - perhaps a shortened title
A continuación se presenta el modelo implementado en SARAH versión 4.9.1. Los archivos presentados en este anexo como los programas implementados en Python se pueden encontrar en el repositorio \github{https://github.com/AgudeloKimy/U1Dark.git}.             
\section{model.m}
\begin{lstlisting}
Model`Name = "Udark";
Model`NameLaTeX ="B-L extended Standard Model";
Model`Authors = "Kimy Agudelo, Diego Restrepo and David Suarez";
Model`Date = "2023-04-16";


(*-------------------------------------------*)
(*   Particle Content*)
(*-------------------------------------------*)

Global[[1]] = {Z[2], Z2};

(* Gauge Superfields *)

Gauge[[1]]={B,   U[1], hypercharge,  g1, False, 1};
Gauge[[2]]={WB, SU[2], left,         g2,  True, 1};
Gauge[[3]]={G,  SU[3], color,        g3, False, 1};
Gauge[[4]]={Bp,  U[1], U1L,         g1p, False, 1};


(* Chiral Superfields *)

FermionFields[[1]]  = { q, 3, {uL, dL},  1/6, 2,  3,    0,  1};  
FermionFields[[2]]  = { l, 3, {vL, eL}, -1/2, 2,  1,    0,  1};
FermionFields[[3]]  = { d, 3, conj[dR],  1/3, 1, -3,    0,  1};
FermionFields[[4]]  = { u, 3, conj[uR], -2/3, 1, -3,    0,  1};
FermionFields[[5]]  = { e, 3, conj[eR],    1, 1,  1,    0,  1};
FermionFields[[6]]  = { v, 2, conj[vR],    0, 1,  1,   -5,  1};
FermionFields[[7]]  = {CL, 1,       cl,    0, 1,  1,    6, -1};
FermionFields[[8]]  = {CR, 1, conj[cr],    0, 1,  1,   -1, -1};

ScalarFields[[1]] = { H, 1, {H0, Hm}, -1/2, 2,  1,   0, 1};
ScalarFields[[2]] = {bi, 1,      BiD,    0, 1,  1,  -5, 1};


(*----------------------------------------------*)
(*   DEFINITION                                 *)
(*----------------------------------------------*)

NameOfStates={GaugeES, EWSB};

(* ----- Before EWSB ----- *)

DEFINITION[GaugeES][Additional]= {
	{LagHC, {AddHC->True}},
	{LagNoHC,{ AddHC->False}}
};

LagNoHC = -(mu2 conj[H].H + 1/2 L1 conj[H].H.conj[H].H + MuP conj[bi].bi + 1/2 L2 conj[bi].bi.conj[bi].bi + L3 conj[bi].bi.conj[H].H);

LagHC = - (+ Yd H.d.q + Ye H.e.l - Yu conj[H].u.q  + Yc bi.CL.CR );

(* Gauge Sector *)

DEFINITION[EWSB][GaugeSector] =
{ 
  {{VB,VWB[3],VBp},{VP,VZ,VZp},ZZ},
  {{VWB[1],VWB[2]},{VWm,conj[VWm]},ZW}
};
        
          	

(* ----- VEVs ---- *)

DEFINITION[EWSB][VEVs]= 
{    {H0, {vH, 1/Sqrt[2]}, {sigmaH, \[ImaginaryI]/Sqrt[2]},{phiH, 1/Sqrt[2]}},
     {BiD,{vX, 1/Sqrt[2]}, {sigmaB, \[ImaginaryI]/Sqrt[2]},{phiB, 1/Sqrt[2]}}     };
 

DEFINITION[EWSB][MatterSector]=   
    {
     {{phiH,phiB},{hh,ZH}},
     {{sigmaH,sigmaB},{Ah,ZA}},
     {{{dL}, {conj[dR]}}, {{DL,Vd}, {DR,Ud}}},
     {{{uL}, {conj[uR]}}, {{UL,Vu}, {UR,Uu}}},
     {{{eL}, {conj[eR]}}, {{EL,Ve}, {ER,Ue}}},
     {{vL,conj[vR]}, {VL,ZM}}
};  


(*------------------------------------------------------*)
(* Dirac-Spinors *)
(*------------------------------------------------------*)

DEFINITION[EWSB][DiracSpinors]={
 Fd ->{  DL, conj[DR]},
 Fe ->{  EL, conj[ER]},
 Fu ->{  UL, conj[UR]},
 Fv ->{  VL, conj[VL]},
 Chi ->{  cl, cr}
};

DEFINITION[EWSB][GaugeES]={
 Fd1 ->{  FdL, 0},
 Fd2 ->{  0, FdR},
 Fu1 ->{  Fu1, 0},
 Fu2 ->{  0, Fu2},
 Fe1 ->{  Fe1, 0},
 Fe2 ->{  0, Fe2}};
 

\end{lstlisting}
\section{parameters.m}
\begin{lstlisting}

ParameterDefinitions = { 

{g1,        { Description -> "Hypercharge-Coupling"}},

{g11p,        {Description -> "Mixed Gauge Coupling 2"}},
{g1p1,        {Description -> "Mixed Gauge Coupling 1"}},
{g1p,       {   Description -> "B-L-Coupling"}},

{MZp,       {   Description -> "Z' mass"}},


{g2,        { Description -> "Left-Coupling"}},
{g3,        { Description -> "Strong-Coupling"}},    
{AlphaS,    {Description -> "Alpha Strong"}},	
{e,         { Description -> "electric charge"}},
{Gf,        { Description -> "Fermi's constant"}},
{aEWinv,    { Description -> "inverse weak coupling constant at mZ"}},
 

{Yu,        { Description -> "Up-Yukawa-Coupling",
			 DependenceNum ->  Sqrt[2]/vH* {{Mass[Fu,1],0,0},
             									{0, Mass[Fu,2],0},
             									{0, 0, Mass[Fu,3]}}}}, 
             									
{Yd,        { Description -> "Down-Yukawa-Coupling",
			  DependenceNum ->  Sqrt[2]/vH* {{Mass[Fd,1],0,0},
             									{0, Mass[Fd,2],0},
             									{0, 0, Mass[Fd,3]}}}},
             									
{Ye,        { Description -> "Lepton-Yukawa-Coupling",
			  DependenceNum ->  Sqrt[2]/vH* {{Mass[Fe,1],0,0},
             									{0, Mass[Fe,2],0},
             									{0, 0, Mass[Fe,3]}}}}, 
                                                                            
                                                                           
{Mu,         { Description -> "SM Mu Parameter"}},                                        
{\[Lambda],  { Description -> "SM Higgs Selfcouplings"}},
{vH,          { Description -> "EW-VEV",
               DependenceSPheno -> None }},
{vX,      {  LaTeX -> "x",
             Dependence ->  None, 
             OutputName -> vX,
             Real -> True,
             LesHouches -> {BL,43} }},

{ThetaW,    { Description -> "Weinberg-Angle"}},
{ThetaWp,  { Description -> "Theta'", DependenceNum -> None  }},

{ZZ, {Description ->   "Photon-Z-Z' Mixing Matrix"}},
{ZW, {Description -> "W Mixing Matrix"}},

{L1, {OutputName -> lam1,
      LaTeX -> "\\lambda_1",
      LesHouches -> {BL,1}}},


{L2, {OutputName -> lam2,
      LaTeX -> "\\lambda_2",
      LesHouches -> {BL,2}}},

{L3, {OutputName -> lam3,
      LaTeX -> "\\lambda_3",
      LesHouches -> {BL,3}}},

{MuP, {OutputName -> MUP,
      LaTeX -> "\\mu'",
      LesHouches -> {BL,10}}},

{mu2, {OutputName -> mu,
      LaTeX -> "\\mu_2",
      LesHouches -> {BL,11}}},

{mH2, {OutputName -> mH2,
      LaTeX -> "\\m^2_H",
      LesHouches -> {BL,20}}},

{mchi2, {OutputName -> mX2,
         LaTeX -> "\\m^2_\\chi",
         LesHouches -> {BL,22}}},
      
{Yx, {OutputName -> YX,
      LaTeX -> "y_x",
      LesHouches -> {BL,31}}},
      
{Yc, {OutputName -> YC,
      LaTeX -> "y_c",
      LesHouches -> {BL,32}}},

{MUS, {OutputName -> muS,
      LaTeX -> "\\mu_s",
      LesHouches -> {BL,30}}},
      
{Vu,        {Description ->"Left-Up-Mixing-Matrix"}},
{Vd,        {Description ->"Left-Down-Mixing-Matrix"}},
{Uu,        {Description ->"Right-Up-Mixing-Matrix"}},
{Ud,        {Description ->"Right-Down-Mixing-Matrix"}}, 
{Ve,        {Description ->"Left-Lepton-Mixing-Matrix"}},
{Ue,        {Description ->"Right-Lepton-Mixing-Matrix"}},

{ZM,	    {Description -> "Neutrino-Mixing-Matrix"}},

{ZH,        { Description->"Scalar-Mixing-Matrix", 
               Dependence -> None,
               DependenceOptional -> None,
               DependenceNum -> None   }},
{ZA,        { Description->"Pseudo-Scalar-Mixing-Matrix", 
                Dependence -> None,
               DependenceOptional -> None,
               DependenceNum -> None   }}



 }; 

\end{lstlisting}

%%%%%%%%%%%%%%%%%%%%%%%%%%%%%%%%%%%%%%%%%%%%%%%%%%%%%%%%%


\section{particles.m}
\begin{lstlisting}

ParticleDefinitions[GaugeES] = {
      {H0,  { 
                 PDG -> 0,
                 Width -> 0, 
                 Mass -> Automatic,
                 FeynArtsNr -> 1,
                 LaTeX -> "H^0",
                 OutputName -> "H0" }},                         
      
      
      {Hp,  { 
                 PDG -> 0,
                 Width -> 0, 
                 Mass -> Automatic,
                 FeynArtsNr -> 2,
                 LaTeX -> "H^+",
                 OutputName -> "Hp" }}, 
                 
               
    
      {VB,   { Description -> "B-Boson"}},                                                   
      {VG,   { Description -> "Gluon"}},          
      {VWB,  { Description -> "W-Bosons"}},          
      {gB,   { Description -> "B-Boson Ghost"}},                                                   
      {gG,   { Description -> "Gluon Ghost" }},          
      {gWB,  { Description -> "W-Boson Ghost"}}
      
      };
      
      
      
      
  ParticleDefinitions[EWSB] = {
            
      
    {hh   ,  { Description -> "Higgs",
                 PDG -> {25,35},
                 Width -> Automatic, 
                 Mass ->LesHouches,
                 FeynArtsNr -> 1,
                 LaTeX -> "h",
                 OutputName -> "h" }}, 
                 
     {Ah   ,  {  Description -> "Pseudo-Scalar Higgs",
                 PDG -> {0,0},
                 Width -> {0, External}, 
                 Mass ->LesHouches,
                 FeynArtsNr -> 2,
                 LaTeX -> "A_h",
                 OutputName -> "Ah" }},                      
      
      
     {Hm,     {   Description -> "Charged Higgs", 
                 PDG -> {0},
                 Width -> 0, 
                 Mass ->LesHouches,
                 FeynArtsNr -> 3,
                 LaTeX -> "H^-",
                 OutputName -> "Hm" }},                                              
      
       {VP,   { Description -> "Photon"}}, 
      {VZ,   { Description -> "Z-Boson",
      			 Goldstone -> Ah[{1}] }}, 
      {VG,   { Description -> "Gluon" }},          
      {VWm,  { Description -> "W-Boson",
                Goldstone->Hm }},         
      {gP,   { Description -> "Photon Ghost"}},                                                   
      {gWm,  { Description -> "Negative W-Boson Ghost"}}, 
      {gWmC, { Description -> "Positive W-Boson Ghost" }}, 
      {gZ,   { Description -> "Z-Boson Ghost" }},
      {gG,   { Description -> "Gluon Ghost" }},          
      {VZp,    { Description -> "Z'-Boson",
      			 Goldstone -> Ah[{2}]}},  
      {gZp,    { Description -> "Z'-Ghost" }},    
                               
                 
      {Fd,   { Description -> "Down-Quarks"}},   
      {Fu,   { Description -> "Up-Quarks"}},   
      {Fe,   { Description -> "Leptons" }},
      {Fv,   { Description -> "Neutrinos",
      			PDG ->{12,14,16,8810012,8810014} }},
          {Chi,  { Description -> "Singlet Fermions",
	       PDG -> {1022},
	       Mass -> LesHouches,
	       ElectricCharge -> 0,
	       LaTeX -> "\\Xi",
	       OutputName -> "Xi" }}      
     
        };    
        
        
        
 WeylFermionAndIndermediate = {
     
    {H,      {   PDG -> 0,
                 Width -> 0, 
                 Mass -> Automatic,
                 LaTeX -> "H",
                 OutputName -> "" }},

   {dR,     {LaTeX -> "d_R" }},
   {eR,     {LaTeX -> "e_R" }},
   {lep,     {LaTeX -> "l" }},
   {uR,     {LaTeX -> "u_R" }},
   {q,     {LaTeX -> "q" }},
   {eL,     {LaTeX -> "e_L" }},
   {dL,     {LaTeX -> "d_L" }},
   {uL,     {LaTeX -> "u_L" }},
   {vL,     {LaTeX -> "\\nu_L" }},

   {DR,     {LaTeX -> "D_R" }},
   {ER,     {LaTeX -> "E_R" }},
   {UR,     {LaTeX -> "U_R" }},
   {EL,     {LaTeX -> "E_L" }},
   {DL,     {LaTeX -> "D_L" }},
   {UL,     {LaTeX -> "U_L" }}
        };       
\end{lstlisting}


%%%%%%%%%%%%%%%%%%%%%%%%%%%%%%%%%%%%%%%%%%%%%%%%%%%%%%%%%


\section{SPheno.m}
\begin{lstlisting}
OnlyLowEnergySPheno = True;


MINPAR={{1,Lambda1INPUT},
        {2,Lambda2INPUT},
        {3,Lambda3INPUT},
        {10, g1pINPUT},
        {11, g1p1INPUT},
        {12, g11pINPUT},
        {20, vXinput},
        {21, Ycinput}        
};

ParametersToSolveTadpoles = {MuP,mu2};

DEFINITION[MatchingConditions]= {
 {Ye, YeSM},
 {Yd, YdSM},
 {Yu, YuSM},
 {g1, g1SM},
 {g2, g2SM},
 {g3, g3SM},
 {vH, vSM}
 };


BoundaryLowScaleInput={
 {g1p,g1pINPUT},
 {g11p,g11pINPUT},
 {g1p1,g1p1INPUT},
 {L1, Lambda1INPUT},
 {L2, Lambda2INPUT},
 {L3, Lambda3INPUT},
 {vX, vXinput},
 {Yc, Ycinput}
};

AddTreeLevelUnitarityLimits=True;

ListDecayParticles = {Fu,Fe,Fd,Fv,hh, VZp, Chi};
ListDecayParticles3B = {{Fv,"Fv.f90"},{Fu,"Fu.f90"},{Fe,"Fe.f90"},{Fd,"Fd.f90"}};
\end{lstlisting}



%%%%%%%%%%%%%%%%%%%%%%%%%%%%%%%%%%%%%%%%%%%%%%%%%%%%%%%%%


\section{LesHouches.in.MDEODARK2}
\begin{lstlisting}
Block MODSEL      #  
 1 1               #  1/0: High/low scale input 
 2 1              # Boundary Condition  
 6 1               # Generation Mixing 
Block SMINPUTS    # Standard Model inputs 
 2 1.166370E-05    # G_F,Fermi constant 
 3 1.187000E-01    # alpha_s(MZ) SM MSbar 
 4 9.118870E+01    # Z-boson pole mass 
 5 4.180000E+00    # m_b(mb) SM MSbar 
 6 1.735000E+02    # m_top(pole) 
 7 1.776690E+00    # m_tau(pole) 
Block MINPAR      # Input parameters 
 1   2.577338E-01    # Lambda1INPUT
 2   4.900000E-01    # Lambda2INPUT
 3   0.000000E+00    # Lambda3INPUT
 10   1.000000E-01    # g1pINPUT
 11   0.000000E+00    # g1p1INPUT
 12   0.000000E+00    # g11pINPUT
 20   1.000000E+03    # vXinput
 21   9.000000E-01    # Ycinpu
Block SPhenoInput   # SPheno specific input 
  1 -1              # error level 
  2  0              # SPA conventions 
  7  0              # Skip 2-loop Higgs corrections 
  8  3              # Method used for two-loop calculation 
  9  1              # Gaugeless limit used at two-loop 
 10  0              # safe-mode used at two-loop 
 11 1               # calculate branching ratios 
 13 1               # 3-Body decays: none (0), fermion (1), scalar (2), both (3) 
 14 0               # Run couplings to scale of decaying particle 
 12 1.000E-04       # write only branching ratios larger than this value 
 15 1.000E-30       # write only decay if width larger than this value 
 19 -2              # Matching order (-2:automatic, -1:pole, 0-2: tree, one- & two-loop) 
 31 -1              # fixed GUT scale (-1: dynamical GUT scale) 
 32 0               # Strict unification 
 34 1.000E-04       # Precision of mass calculation 
 35 40              # Maximal number of iterations
 36 5               # Minimal number of iterations before discarding points
 37 1               # Set Yukawa scheme  
 38 2               # 1- or 2-Loop RGEs 
 50 1               # Majorana phases: use only positive masses (put 0 to use file with CalcHep/Micromegas!) 
 51 0               # Write Output in CKM basis 
 52 0               # Write spectrum in case of tachyonic states 
 55 0               # Calculate loop corrected masses 
 57 1               # Calculate low energy constraints 
 60 1               # Include possible, kinetic mixing 
 65 1               # Solution tadpole equation 
 66 1               # Two-Scale Matching 
 67 1               # effective Higgs mass calculation 
 75 0               # Write WHIZARD files 
 76 2               # Write HiggsBounds file: 2 for HiggsBounds5, 1 for HiggsBounds4 and below   
 77 0               # Output for MicrOmegas (running masses for light quarks; real mixing matrices)   
 78 0               # Output for MadGraph (writes also vanishing blocks)   
 79 0               # Write WCXF files (exchange format for Wilson coefficients) 
 86 0.              # Maximal width to be counted as invisible in Higgs decays; -1: only LSP 
 440 1               # Tree-level unitarity constraints (limit s->infinity) 
 441 1               # Full tree-level unitarity constraints 
 442 1000.           # sqrt(s_min)   
 443 2000.           # sqrt(s_max)   
 444 5               # steps   
 445 0               # running   
 510 0.              # Write tree level values for tadpole solutions 
 515 0               # Write parameter values at GUT scale 
 520 1.              # Write effective Higgs couplings (HiggsBounds blocks): put 0 to use file with MadGraph! 
 521 1.              # Diphoton/Digluon widths including higher order 
 525 0.              # Write loop contributions to diphoton decay of Higgs 
 530 0.              # Write Blocks for Vevacious 
Block DECAYOPTIONS   # Options to turn on/off specific decays 
1    1     # Calc 3-Body decays of Fv 
2    1     # Calc 3-Body decays of Fu 
3    1     # Calc 3-Body decays of Fe 
4    1     # Calc 3-Body decays of Fd 
\end{lstlisting}
